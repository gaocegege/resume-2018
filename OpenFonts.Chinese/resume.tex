%%%%%%%%%%%%%%%%%%%%%%%%%%%%%%%%%%%%%%%
% Deedy - One Page Two Column Resume
% LaTeX Template
% Version 1.2 (16/9/2014)
%
% Original author:
% Debarghya Das (http://debarghyadas.com)
%
% Original repository:
% https://github.com/deedydas/Deedy-Resume
%
% IMPORTANT: THIS TEMPLATE NEEDS TO BE COMPILED WITH XeLaTeX
%
% This template uses several fonts not included with Windows/Linux by
% default. If you get compilation errors saying a font is missing, find the line
% on which the font is used and either change it to a font included with your
% operating system or comment the line out to use the default font.
% 
%%%%%%%%%%%%%%%%%%%%%%%%%%%%%%%%%%%%%%
% 
% TODO:
% 1. Integrate biber/bibtex for article citation under publications.
% 2. Figure out a smoother way for the document to flow onto the next page.
% 3. Add styling information for a "Projects/Hacks" section.
% 4. Add location/address information
% 5. Merge OpenFont and MacFonts as a single sty with options.
% 
%%%%%%%%%%%%%%%%%%%%%%%%%%%%%%%%%%%%%%
%
% CHANGELOG:
% v1.1:
% 1. Fixed several compilation bugs with \renewcommand
% 2. Got Open-source fonts (Windows/Linux support)
% 3. Added Last Updated
% 4. Move Title styling into .sty
% 5. Commented .sty file.
%
%%%%%%%%%%%%%%%%%%%%%%%%%%%%%%%%%%%%%%%
%
% Known Issues:
% 1. Overflows onto second page if any column's contents are more than the
% vertical limit
% 2. Hacky space on the first bullet point on the second column.
%
%%%%%%%%%%%%%%%%%%%%%%%%%%%%%%%%%%%%%%


\documentclass[]{deedy-resume-openfont}
\usepackage{fancyhdr}
    
\pagestyle{fancy}
\fancyhf{}
    
\begin{document}

%%%%%%%%%%%%%%%%%%%%%%%%%%%%%%%%%%%%%%
%
%     LAST UPDATED DATE
%
%%%%%%%%%%%%%%%%%%%%%%%%%%%%%%%%%%%%%%
\lastupdated

%%%%%%%%%%%%%%%%%%%%%%%%%%%%%%%%%%%%%%
%
%     TITLE NAME
%
%%%%%%%%%%%%%%%%%%%%%%%%%%%%%%%%%%%%%%
\namesection{高}{策}{ \urlstyle{same}\href{mailto:ce.gao@outlook.com}{ce.gao@outlook.com} | 1592 1592 066
}

%%%%%%%%%%%%%%%%%%%%%%%%%%%%%%%%%%%%%%
%
%     COLUMN ONE
%
%%%%%%%%%%%%%%%%%%%%%%%%%%%%%%%%%%%%%%

\begin{minipage}[t]{0.2\textwidth} 

%%%%%%%%%%%%%%%%%%%%%%%%%%%%%%%%%%%%%%
%     EDUCATION
%%%%%%%%%%%%%%%%%%%%%%%%%%%%%%%%%%%%%%

\section{教育经历} 
\sectionsep

\subsection{上海交通大学}
\descript{硕士学位,主修软件工程}
\location{2016.09-2019.03}
\sectionsep

\subsection{上海交通大学}
\descript{学士学位,主修软件工程}
\descript{辅修法学}
\location{2012.09-2016.09}
\sectionsep

%%%%%%%%%%%%%%%%%%%%%%%%%%%%%%%%%%%%%%
%     LINKS
%%%%%%%%%%%%%%%%%%%%%%%%%%%%%%%%%%%%%%

\section{相关链接}
\sectionsep
Blog://  \href{http://gaocegege.com/Blog}{gaocegege.com/Blog} \\
{(总计\textbf{8万}阅读,\textbf{3万}访客)} \\    
Github:// \href{https://github.com/gaocegege}{gaocegege} \\
{(\textbf{ 420+ }关注者)} \\
LinkedIn://  \href{https://www.linkedin.com/in/gaocegege}{gaocegege} \\

\section{技术分享}
\sectionsep
\href{http://slides.com/gaocegege/processing-r}{Processing.R} @ \href{http://china-r.org/sh2017/index.html}{第十届中国R会议}

%%%%%%%%%%%%%%%%%%%%%%%%%%%%%%%%%%%%%%
%     COURSEWORK
%%%%%%%%%%%%%%%%%%%%%%%%%%%%%%%%%%%%%%

% \section{修读课程}
% \subsection{Graduate}
% Advanced Machine Learning \\
% Open Source Software Engineering \\
% Advanced Interactive Graphics \\
% Compilers + Practicum \\
% Cloud Computing \\
% Evolutionary Computation \\
% Defending Computer Networks \\
% Machine Learning \\
% \sectionsep

%%%%%%%%%%%%%%%%%%%%%%%%%%%%%%%%%%%%%%
%     SKILLS
%%%%%%%%%%%%%%%%%%%%%%%%%%%%%%%%%%%%%%

\section{编程技能}
\sectionsep
\subsection{编程语言}
\location{超过 5000 行}
Go \textbullet{} Python \textbullet{} C++ \textbullet{} Java \\
\location{1000 - 5000 行}
R \textbullet{} C \textbullet{} Scala \textbullet{} \LaTeX\ \\
\sectionsep

\subsection{云计算}
\location{一般}
Docker \textbullet{} Kubernetes \\
\sectionsep

\subsection{DevOps}
\location{一般}
微服务 \textbullet{} Jenkins \textbullet{} Travis CI  \\
\sectionsep
%%%%%%%%%%%%%%%%%%%%%%%%%%%%%%%%%%%%%%
%
%     COLUMN TWO
%
%%%%%%%%%%%%%%%%%%%%%%%%%%%%%%%%%%%%%%

\end{minipage} 
\hfill
\begin{minipage}[t]{0.78\textwidth} 

%%%%%%%%%%%%%%%%%%%%%%%%%%%%%%%%%%%%%%
%     EXPERIENCE
%%%%%%%%%%%%%%%%%%%%%%%%%%%%%%%%%%%%%%

\section{实习经历}
\sectionsep
\runsubsection{谷歌编程之夏}
\descript{学生参与者}
\location{2017.05 - 2017.09 | 远程}
\vspace{\topsep}
\begin{tightemize}
    \item 共有 20651 个注册学生,其中 1318 个申请被谷歌接收,\textbf{接收率 6\%}
    \item 为 Processing 基金会实现了对 Processing 的 R 语言支持
    \item 与社区紧密合作,实现对 Processing 库的支持和对 R 包的支持
    \item 所做项目 \href{https://github.com/gaocegege/Processing.R}{\bf Processing.R} 在 GitHub 上获得 \textbf{70 stars},成为本次编程之夏 star 最多的项目
\end{tightemize}
\sectionsep

\runsubsection{摩根士丹利}
\descript{CIP 项目实习生}
\location{2017.02-2017.08 | 上海}
\begin{tightemize}
\item 优化开源容器调度管理框架 treadmill 的调度器
\item 实现与 Kubernetes 类似的调度模型,同时保留自身的树形结构
\end{tightemize}
\sectionsep

\runsubsection{上海触宝信息技术有限公司}
\descript{数据工程师(实习)}
\location{2015.09-2015.09 | 上海}
\begin{tightemize}
\item 移植爬虫代码到新的平台,优化重写部分过期的爬虫
\end{tightemize}
\sectionsep

\runsubsection{蚂蚁金服(杭州)网络技术有限公司}
\descript{Java 研发工程师(实习)| 杭州}
\location{2015.07-2015.09}
\begin{tightemize}
\item 在支付宝国际事业团队从事海外直购业务开发
\item 实现部分包裹清关的逻辑和后台管理的逻辑
\end{tightemize}
\sectionsep

%%%%%%%%%%%%%%%%%%%%%%%%%%%%%%%%%%%%%%
%     RESEARCH
%%%%%%%%%%%%%%%%%%%%%%%%%%%%%%%%%%%%%%

\section{项目与论文}
\sectionsep

\runsubsection{\href{https://github.com/caicloud/cyclone}{Cyclone}}
\descript{Maintainer}
\location{2016.11}
\begin{tightemize}
    \item 基于 Docker 的持续集成与持续部署系统,在 GitHub 上获得 \textbf{440 stars}
    \item 确定架构选型,实现 YAML 配置解析和 Docker 的运行时集成
    \end{tightemize}
\sectionsep

\runsubsection{\href{https://github.com/gaocegege/scrala}{Scrala}}
\descript{Owner}
\location{2015.12}
\begin{tightemize}
    \item 使用 scala 实现的爬虫框架,灵感来自 scrapy
    \item 在 GitHub 上获得 \textbf{70 stars}
    \item 底层使用 Actor 模型取代 Python 中的异步模型
    \end{tightemize}
\sectionsep

\runsubsection{\href{https://github.com/prism-river/killy}{Killy}}
\descript{Owner}
\location{2017.10}
\begin{tightemize}
    \item 在 Minecraft 中查看 TiDB 集群状态,获得 Go Hack 17 二等奖以及 PingCAP 专项奖
    \end{tightemize}
\sectionsep

%%%%%%%%%%%%%%%%%%%%%%%%%%%%%%%%%%%%%%
%     OPEN SOURCE
%%%%%%%%%%%%%%%%%%%%%%%%%%%%%%%%%%%%%%

\section{开源贡献}
\begin{tabular}{ll}
\href{https://github.com/moby/moby/commits?author=gaocegege}{\bf moby/moby} & 实现 docker service ps -q 参数,与 swarmkit 更好集成 \\
\href{https://github.com/opencontainers/runc/commits?author=gaocegege}{\bf opencontainers/runc} & 为了修复 \href{https://github.com/moby/moby/issues/27484}{moby/moby\#27484} 对上游进行的修改 \\
\href{https://github.com/pingcap/tidb/commits?author=gaocegege}{\bf pingcap/tidb} & 在 travis 里引入了覆盖率测试; 实现 truncate 函数 \\
\href{https://github.com/coala/coala-vs-code/commits/master?author=gaocegege}{coala/coala-vs-code} & Visual Studio Code 上的插件,项目 maintainer \\
\href{https://github.com/weijianwen/SJTUThesis/commits?author=gaocegege}{weijianwen/SJTUThesis} & 为学士论文模板添加英文大摘要; 替换版权字体 \\
\end{tabular}
\sectionsep

%%%%%%%%%%%%%%%%%%%%%%%%%%%%%%%%%%%%%%
%     AWARDS
%%%%%%%%%%%%%%%%%%%%%%%%%%%%%%%%%%%%%%

\section{所获奖项} 
\begin{tabular}{rll}
% 2017.10     & 二等奖  & Go 基金会中国黑客马拉松 \\
% 2017.10     & PingCAP 专项奖 & Go 基金会中国黑客马拉松 \\
2017.10     & 奖学金  & 因特尔中国奖学金计划 \\
2016.11	    & 一等奖  & 第七届中国大学生服务外包创新创业大赛 \\
2016.09	    & 二等奖  & 第十三届全国研究生数学建模竞赛 \\
% 2015.03	    & 二等奖  & 美国大学生数学建模竞赛 \\
% 2014.11     & 一等奖 & 中国大学生数学建模竞赛上海赛区 \\
% 2014.07	    & 二等奖/杰出个人  & 大众点评校园 Hackathon \\
\end{tabular}
\sectionsep

%%%%%%%%%%%%%%%%%%%%%%%%%%%%%%%%%%%%%%
%     PUBLICATIONS
%%%%%%%%%%%%%%%%%%%%%%%%%%%%%%%%%%%%%%

% \section{Publications} 
% \renewcommand\refname{\vskip -1.5cm} % Couldn't get this working from the .cls file
% \bibliographystyle{abbrv}
% \bibliography{publications}
% \nocite{*}

\end{minipage} 
\end{document}  \documentclass[]{article}
