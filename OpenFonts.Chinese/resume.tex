%%%%%%%%%%%%%%%%%%%%%%%%%%%%%%%%%%%%%%%
% Deedy - One Page Two Column Resume
% LaTeX Template
% Version 1.2 (16/9/2014)
%
% Original author:
% Debarghya Das (http://debarghyadas.com)
%
% Original repository:
% https://github.com/deedydas/Deedy-Resume
%
% IMPORTANT: THIS TEMPLATE NEEDS TO BE COMPILED WITH XeLaTeX
%
% This template uses several fonts not included with Windows/Linux by
% default. If you get compilation errors saying a font is missing, find the line
% on which the font is used and either change it to a font included with your
% operating system or comment the line out to use the default font.
%
%%%%%%%%%%%%%%%%%%%%%%%%%%%%%%%%%%%%%%
%
% TODO:
% 1. Integrate biber/bibtex for article citation under publications.
% 2. Figure out a smoother way for the document to flow onto the next page.
% 3. Add styling information for a "Projects/Hacks" section.
% 4. Add location/address information
% 5. Merge OpenFont and MacFonts as a single sty with options.
%
%%%%%%%%%%%%%%%%%%%%%%%%%%%%%%%%%%%%%%
%
% CHANGELOG:
% v1.1:
% 1. Fixed several compilation bugs with \renewcommand
% 2. Got Open-source fonts (Windows/Linux support)
% 3. Added Last Updated
% 4. Move Title styling into .sty
% 5. Commented .sty file.
%
%%%%%%%%%%%%%%%%%%%%%%%%%%%%%%%%%%%%%%%
%
% Known Issues:
% 1. Overflows onto second page if any column's contents are more than the
% vertical limit
% 2. Hacky space on the first bullet point on the second column.
%
%%%%%%%%%%%%%%%%%%%%%%%%%%%%%%%%%%%%%%


\documentclass[]{deedy-resume-openfont}
\usepackage{fancyhdr}

\pagestyle{fancy}
\fancyhf{}

\begin{document}

%%%%%%%%%%%%%%%%%%%%%%%%%%%%%%%%%%%%%%
%
%     LAST UPDATED DATE
%
%%%%%%%%%%%%%%%%%%%%%%%%%%%%%%%%%%%%%%
\lastupdated

\namesection{高}{策}{ \urlstyle{same}\href{http://gaocegege.com}{gaocegege.com}| 1592 1592 066 | \href{mailto:ce.gao@outlook.com}{ce.gao@outlook.com} \\
{寻找 2019 SDE/SRE/AI infra 全职工作}
}

%%%%%%%%%%%%%%%%%%%%%%%%%%%%%%%%%%%%%%
%
%     COLUMN ONE
%
%%%%%%%%%%%%%%%%%%%%%%%%%%%%%%%%%%%%%%

\begin{minipage}[t]{0.3\textwidth}

%%%%%%%%%%%%%%%%%%%%%%%%%%%%%%%%%%%%%%
%     EDUCATION
%%%%%%%%%%%%%%%%%%%%%%%%%%%%%%%%%%%%%%

\section{教育经历}
\sectionsep

\subsection{上海交通大学}
\descript{工学硕士学位,软件工程}
\descript{GPA 3.68/4.0}
\location{2016.09-2019.03}
\sectionsep

\subsection{上海交通大学}
\descript{工学学士学位,软件工程}
\location{2012.09-2016.09}
\sectionsep

\subsection{上海交通大学}
\descript{第二专业法学学士学位,法律}
\location{2014.02-2016.09}
\sectionsep

%%%%%%%%%%%%%%%%%%%%%%%%%%%%%%%%%%%%%%
%     LINKS
%%%%%%%%%%%%%%%%%%%%%%%%%%%%%%%%%%%%%%

\section{相关链接}
\sectionsep
Blog://  \href{http://gaocegege.com/Blog}{gaocegege.com/Blog} \\
{(总计\textbf{ 100k }阅读,\textbf{ 40k }访客)} \\
Github:// \href{https://github.com/gaocegege}{gaocegege} \\
{(\textbf{ 656 }关注者)} \\
LinkedIn://  \href{https://www.linkedin.com/in/gaocegege}{gaocegege} \\
知乎:// \href{https://www.zhihu.com/people/gaocegege}{gaocegege} \\
\sectionsep

\section{部分博文}
\sectionsep
\href{http://gaocegege.com/Blog/kubernetes/operator}{Kubernetes CRD Operator 实现指南} \\
\href{http://gaocegege.com/Blog/%E6%9C%BA%E5%99%A8%E5%AD%A6%E4%B9%A0/katib}{Katib: Kubernetes native 的超参数训练系统} \\
\href{http://gaocegege.com/Blog/ml%20system/kubeflow}{Kubeflow: 在 Kubernetes 上进行机器学习} \\
\href{http://gaocegege.com/Blog/%E6%BA%90%E7%A0%81%E5%88%86%E6%9E%90/kubernetes-scheduler}{浅入了解容器编排框架调度器之 Kubernetes} \\
\href{http://gaocegege.com/Blog/%E9%9A%8F%E7%AC%94/consistency}{小议分布式系统的一致性模型} \\
\sectionsep

\section{技术分享}
\sectionsep
\href{https://github.com/gaocegege/papers-notebook}{论文阅读笔记} @ GitHub \\
\href{https://docs.google.com/presentation/d/1ED24TCnlBVzyJz0aCEAtXQQh0_W1RKSeapP3QZ0fTKA/edit?usp=sharing}{Kubeflow} @ 统计之都 Meetup \\
\href{http://slides.com/gaocegege/processing-r}{Processing.R} @ \href{http://china-r.org/sh2017/index.html}{第十届 R 语言会议} \\
\href{https://docs.google.com/presentation/d/1ylRT4VvydWbR7SyTQzNZOLpkXtgSZJiEl5nmXY1KuJw/edit?usp=sharing}{Processing.R} @ \href{https://zhuanlan.zhihu.com/dongyue}{东岳} 2017 技术分享 \\
\href{https://docs.google.com/presentation/d/1Ru4Dm9TLoyxnJgFqvsCHrb82VT622H-zBSgAe1vJL44/edit?usp=sharing}{Docker 介绍}  @ \href{https://zhuanlan.zhihu.com/dongyue}{东岳} 2016 技术分享 \\
\sectionsep

\section{语言能力}
\sectionsep
英语六级 531 \\
\sectionsep

%%%%%%%%%%%%%%%%%%%%%%%%%%%%%%%%%%%%%%
%     COURSEWORK
%%%%%%%%%%%%%%%%%%%%%%%%%%%%%%%%%%%%%%

% \section{修读课程}
% \subsection{Graduate}
% Advanced Machine Learning \\
% Open Source Software Engineering \\
% Advanced Interactive Graphics \\
% Compilers + Practicum \\
% Cloud Computing \\
% Evolutionary Computation \\
% Defending Computer Networks \\
% Machine Learning \\
% \sectionsep

%%%%%%%%%%%%%%%%%%%%%%%%%%%%%%%%%%%%%%
%
%     COLUMN TWO
%
%%%%%%%%%%%%%%%%%%%%%%%%%%%%%%%%%%%%%%

\end{minipage}
\hfill
\begin{minipage}[t]{0.68\textwidth}

%%%%%%%%%%%%%%%%%%%%%%%%%%%%%%%%%%%%%%
%     EXPERIENCE
%%%%%%%%%%%%%%%%%%%%%%%%%%%%%%%%%%%%%%

\section{实习经历}

\sectionsep
\runsubsection{才云科技}
\descript{合作研究学生}
\location{2017.12 - Now | 远程}
\vspace{\topsep}
\begin{tightemize}
    \item 为 \href{https://github.com/kubeflow/kubeflow}{Kubeflow} 社区维护 TensorFlow 分布式训练支持 \href{https://github.com/kubeflow/tf-operator}{kubeflow/tf-operator} 和超参数训练系统 \href{https://github.com/kubeflow/katib}{kubeflow/katib}
    \item 分析优化分布式机器学习任务在大规模机器集群上的调度
\end{tightemize}
\sectionsep

\sectionsep
\runsubsection{coala 社区}
\descript{Google Summer of Code 2018 导师}
\location{2018.03 - Now | 远程}
\vspace{\topsep}
\begin{tightemize}
    \item 指导来自印度的一位学生参与 Google Summer of Code 2018 项目,为 coala 社区完善 Language Server 实现,并在一个编辑器客户端上验证功能
\end{tightemize}
\sectionsep

\sectionsep
\runsubsection{coala 社区}
\descript{Google Code In 2018 导师}
\location{2017.12 - 2018.02 | 远程}
\vspace{\topsep}
\begin{tightemize}
    \item 基于 Language Server Protocol 实现 coala 在 Visual Studio Code 上的插件,共被下载使用 \textbf{460} 次
    \item 担任 Google Code In 2018 导师,指导来自全球的高中生参与开源项目
\end{tightemize}
\sectionsep

\sectionsep
\runsubsection{Processing 基金会}
\descript{Google Summer of Code 2017 学生参与者}
\location{2017.05 - 2017.09 | 远程}
\vspace{\topsep}
\begin{tightemize}
    \item 本次 GSoC \textbf{接收率 6\%}(1318/20651)
    \item 为 Processing 基金会实现了对 Processing 的 R 语言支持
    \item 所做项目 \href{https://github.com/gaocegege/Processing.R}{\bf Processing.R} 获得 \textbf{92 stars},成为本次编程之夏 star 最多的项目
\end{tightemize}
\sectionsep

\runsubsection{摩根士丹利}
\descript{项目实习生}
\location{2017.02-2017.08 | 上海}
\begin{tightemize}
\item 为开源容器调度管理框架 treadmill 实现与 Kubernetes 类似的调度模型
\item 调度延迟在 100 节点规模下与原本的调度器相比下降 12\% 但增强了其可配置性,支持对节点上硬件资源的动态监视
\end{tightemize}
\sectionsep

\runsubsection{上海触宝信息技术有限公司}
\descript{数据工程师(实习)}
\location{2015.09 - 2015.09 | 上海}
\begin{tightemize}
\item 移植爬虫代码到新的平台,优化重写部分过期的爬虫
\end{tightemize}
\sectionsep

\runsubsection{蚂蚁金服(杭州)网络技术有限公司}
\descript{Java 研发工程师(实习)}
\location{2015.07 - 2015.09 | 杭州}
\begin{tightemize}
\item 在支付宝国际事业部创新业务组任职,从事海外直购业务开发
\item 实现部分包裹清关的逻辑和后台管理的逻辑
\end{tightemize}
\sectionsep

%%%%%%%%%%%%%%%%%%%%%%%%%%%%%%%%%%%%%%
%     RESEARCH
%%%%%%%%%%%%%%%%%%%%%%%%%%%%%%%%%%%%%%

\section{项目与论文}
\sectionsep

\runsubsection{\href{https://github.com/kubeflow/kubeflow}{Kubeflow}}
\descript{Maintainer}
\location{2017.12}
\begin{tightemize}
    \item 在 Kubernetes 上运行 ML 任务,在 GitHub 上有 \textbf{4000 stars}
    \item 维护重构 \href{https://github.com/tensorflow/k8s}{tensorflow/k8s}
    \end{tightemize}
\sectionsep

%%%%%%%%%%%%%%%%%%%%%%%%%%%%%%%%%%%%%%
%     PUBLICATIONS
%%%%%%%%%%%%%%%%%%%%%%%%%%%%%%%%%%%%%%

% \section{Publications}
% \renewcommand\refname{\vskip -1.5cm} % Couldn't get this working from the .cls file
% \bibliographystyle{abbrv}
% \bibliography{publications}
% \nocite{*}

\end{minipage}

\newpage
\pagestyle{fancy}
\fancyhf{}

% PAGE TWO

\begin{minipage}[t]{0.3\textwidth}

%%%%%%%%%%%%%%%%%%%%%%%%%%%%%%%%%%%%%%
%     SKILLS
%%%%%%%%%%%%%%%%%%%%%%%%%%%%%%%%%%%%%%

\section{编程技能}
\sectionsep
\subsection{编程语言}
\location{超过 5000 行}
Go \textbullet{} Python \textbullet{} C++ \textbullet{} Java \\
\location{1000 - 5000 行}
R \textbullet{} C \textbullet{} Scala \textbullet{} \LaTeX\ \\
\sectionsep

\subsection{云计算}
\location{一般}
Docker \textbullet{} Kubernetes \textbullet{} Kubeflow \\
\sectionsep

\subsection{DevOps}
\location{一般}
微服务 \textbullet{} Jenkins \textbullet{} Travis CI  \\
\sectionsep

\end{minipage}
\hfill
\begin{minipage}[t]{0.68\textwidth}

%%%%%%%%%%%%%%%%%%%%%%%%%%%%%%%%%%%%%%
%     OPEN SOURCE
%%%%%%%%%%%%%%%%%%%%%%%%%%%%%%%%%%%%%%

%%%%%%%%%%%%%%%%%%%%%%%%%%%%%%%%%%%%%%
%     AWARDS
%%%%%%%%%%%%%%%%%%%%%%%%%%%%%%%%%%%%%%

\runsubsection{\href{https://github.com/caicloud/cyclone}{Cyclone}}
\descript{Owner}
\location{2016.11}
\begin{tightemize}
    \item 基于 Docker 的持续集成与持续部署系统,在 GitHub 上获得 \textbf{550 stars}
    \item 确定架构选型,实现 YAML 配置解析和 Docker 的运行时集成
    \end{tightemize}
\sectionsep

\runsubsection{\href{https://github.com/gaocegege/scrala}{Scrala}}
\descript{Owner}
\location{2015.12}
\begin{tightemize}
    \item 使用 scala 实现的爬虫框架,灵感来自 scrapy, 在 GitHub 上获得 \textbf{80 stars}
    \item 底层使用 Actor 模型取代 Python 中的异步模型
    \end{tightemize}
\sectionsep

\section{所获奖项}
\sectionsep
\begin{tabular}{rll}
2017.10     & \textbf{二等奖}  & \textbf{Go 基金会中国黑客马拉松} \\
2017.10     & PingCAP 专项奖 & Go 基金会中国黑客马拉松 \\
2017.10     & \textbf{奖学金}  & \textbf{因特尔中国奖学金计划} \\
2016.11	    & 一等奖  & 第七届中国大学生服务外包创新创业大赛 \\
2016.09	    & 二等奖  & 第十三届全国研究生数学建模竞赛 \\
2015.03	    & 二等奖  & 美国大学生数学建模竞赛 \\
2014.11     & 一等奖 & 中国大学生数学建模竞赛上海赛区 \\
2014.07	    & 二等奖/杰出个人  & 大众点评校园 Hackathon \\
2013.09     & 三等奖 & 上海交通大学奖学金 \\
\end{tabular}
\sectionsep

\section{开源贡献}
\sectionsep
\begin{tabular}{ll}
\href{https://github.com/kubeflow/tf-operator/commits?author=gaocegege}{\bf kubeflow/tf-operator} & Maintainer,实现 v1alpha2 版本 \\
\href{https://github.com/kubeflow/katib/commits?author=gaocegege}{\bf kubeflow/katib} & Maintainer,重构代码 \\
\href{https://github.com/moby/moby/commits?author=gaocegege}{\bf moby/moby} & 实现 docker service ps -q 参数 \\
\href{https://github.com/opencontainers/runc/commits?author=gaocegege}{\bf opencontainers/runc} & 为了修复 \href{https://github.com/moby/moby/issues/27484}{moby/moby\#27484} 对上游进行的修改 \\
\href{https://github.com/pingcap/tidb/commits?author=gaocegege}{\bf pingcap/tidb} & 在 travis 里引入了覆盖率测试; 实现 truncate 函数 \\
\href{https://github.com/alibaba/pouch/commits?author=gaocegege}{alibaba/pouch} & 为其 CI 更新 go 版本, 移除对环境路径的 hack \\
\href{https://github.com/hyperhq/runv/commits?author=gaocegege}{hyperhq/runv} & 修复 CLI 描述 \\
\href{https://github.com/kubernetes-incubator/kube-arbitrator/commits?author=gaocegege}{kubernetes/kube-arbitrator} & 引入 go lint verify 测试\\
\href{https://github.com/cncf/devstats/commits?author=gaocegege}{cncf/devstats} & 为 CNCF 基金会的贡献统计系统引入 vendor 管理 \\
\href{https://github.com/kubernetes/kubernetes/commits?author=gaocegege}{kubernetes/kubernetes} & 修复 scheduler 中过期 link \\
\href{https://github.com/prism-river/killy}{prism-river/killy} & Maintainer, Play TiDB in Minecraft \\
\href{https://github.com/coala/coala-vs-code/commits/master?author=gaocegege}{coala/coala-vs-code} & Maintainer,实现 VS Code 插件 \\
\href{https://github.com/gaocegege/maintainer}{Maintainer} & Owner,帮助开发者在 GitHub 上维护项目的工具 \\
\href{https://github.com/siglt/tosknight}{tosknight} & Owner,获得腾讯等网站用户协议并记录版本变化 \\
\href{https://github.com/sjtug/SJTUThesis/commits?author=gaocegege}{sjtug/SJTUThesis} & Maintainer,维护模板,发布版本 \\
\href{https://github.com/dyweb/Deedy-Resume-for-Chinese}{Deedy-Resume-for-Chinese} & Owner,面向本科生的简历论文模板 \\
\href{https://github.com/dyweb/electsys-safari}{electsys-safari} & Owner,上海交大选课插件在 Safari 上的实现 \\
\href{https://github.com/gsoc-cn/gsoc-cn}{gsoc-cn} & Owner,Google Summer of Code 中国社区 \\
更多请见 \href{https://github.com/gaocegege}{gaocegege @ GitHub} & \\
\end{tabular}
\sectionsep

\section{参与活动}
\sectionsep
\sectionsep
\begin{tightemize}
    \item Go Hack 2017 黑客马拉松参赛者:\href{http://gaocegege.com/Blog/%E9%9A%8F%E7%AC%94/killy}{Go Hack 17: Killy 参赛日记}
    \item LinuxCon 17 参会者:\href{http://gaocegege.com/Blog/%E9%9A%8F%E7%AC%94/linuxcon}{LinuxCon 会议回顾}
    \item 2016 年云赛空间黑客马拉松参赛者
    \item 2016 年中国容器大会上海站参会者
    \item 2016 CCTC 云计算会议参会者
    \item 2016 年中国容器大会北京站参会者
    \item HackShanghai 2015 黑客马拉松参赛者
    \item Apache 路演 2015 北京站参会者
    \item HackShanghai 2014 黑客马拉松参赛者:\href{http://newspaper.jfdaily.com/xwcb/html/2014-11/17/content_37290.htm}{新闻晨报报道}
\end{tightemize}
\sectionsep

\sectionsep\end{minipage}

\end{document}
\documentclass[]{article}
